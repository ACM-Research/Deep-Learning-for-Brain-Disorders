% Created 2021-11-11 Thu 01:44
% Intended LaTeX compiler: pdflatex
\documentclass[11pt]{article}
\usepackage[utf8]{inputenc}
\usepackage[T1]{fontenc}
\usepackage{graphicx}
\usepackage{grffile}
\usepackage{longtable}
\usepackage{wrapfig}
\usepackage{rotating}
\usepackage[normalem]{ulem}
\usepackage{amsmath}
\usepackage{textcomp}
\usepackage{amssymb}
\usepackage{capt-of}
\usepackage{hyperref}
\author{Logan Jackson}
\date{\today}
\title{TASKS}
\hypersetup{
 pdfauthor={Logan Jackson},
 pdftitle={TASKS},
 pdfkeywords={},
 pdfsubject={},
 pdfcreator={Emacs 27.2 (Org mode 9.5)}, 
 pdflang={English}}
\begin{document}

\maketitle
\tableofcontents



\section{Alzheimer's}
\label{sec:org4ad5476}

Alzheimer’s disease is the most common type of dementia that affects people over 65 years old.
It is characterized by memory problems or mild cognitive impairment that worsens over time.
In this task we attempt a multimodal classification of different stages of Alzheimer's: Mild Demented,
Moderate Demented, Non Demented, and Very Demented.

For this particular task a CNN is, used the model takes 2D brain data represented with a 4 tensor of shape \((B, H, W, C)\)
where both \(B=1\) and \(C=1\) the reason to use a 4 tensor in this case is because the 2D Convolutional function takes a 4 tensor input.

The convolutional part of the network is then run to extract feature information from the image, after this
the output of the last convolutional layer is, flattened and passed to a fully connected layer in order to map
to 4 different output values (Mild Demented, Moderate Demented, Non Demented, and Very Demented).

\section{Brain Tumors}
\label{sec:org3948501}

Brain tumors are a mass or growth of abnormal, cancerous cells that begin to grow within the brain. For this task
we thought that classification on it's own was not as useful as localization of the tumor, this is because knowing
where and how big the tumor is, is more important to treatment than weather or not they have a tumor in the because
you probably already know someone has a tumor it's one of the easier things to detect in the brain by a human.

But because of the goals of the project a classification model was, written anyways this is because we want to combine
the predictions of the disorders, also if the model is highly confident that there is a brain tumor, it can then run
segmentation on the brain to locate the tumor automatically (very convenient).

The way the classification model works is the same as Alzheimer's, but instead of mapping to 4 classes you only map to 1
making this a binary classification problem, this is probably why the Brain Tumor classification model was so highly accurate.

For segmentation we use U-Net which is, discussed more in the Models section.

\section{Schizophrenia}
\label{sec:orga00695b}

\textbf{\textbf{Some sort of definition or something needs to be, written}}

In the Schizophrenia task domain experts have already preformed feature extraction, on MRI scans of patients with
schizophrenia using AAL Atlas, as well as ``spatial independent component analysis, sliding time
window correlation, and k-means clustering of windowed correlation matrices.'' [1] in order to extract FNC data, ``Functional Network
Connectivity (FNC) are correlation values that summarize the overall connection between independent brain maps over time'' [1]
we use FNC data in order to predict weather or not a patient has Schizophrenia. There were many models tested for this task but the
model with the best performance was XGBoost which is, covered more in the Models section.

\section{Autism Spectrum Disorder (ASD)}
\label{sec:org579ad94}

\textbf{\textbf{Some sort of definition or something needs to be, written}}

For Autism we take in 4D brain data the same type of feature extraction as Schizophrenia is, preformed except we're not professionals,
how this is, done is first segmenting the brain with an atlas then calculating the sparse inverse covariance of brain regions over time
to determine Functional Network Connectivity, this can be improved by parcelating ROIs in the brain based on various criteria then preforming
signal extraction the same way as before.

After the data is, extracted we use an SVM on the extracted data from Normal brains and brains with ASD. SVM is covered more in the models section
\end{document}
